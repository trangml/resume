\subsection{{Autonomous Systems Engineer / Intelligence, Surveillance, and Reconnaissance Division \hfill Mar 2023 --- Present}}
\subtext{MIT Lincoln Laboratory \hfill Lexington, MA}
\begin{zitemize}
	% \item Developing navigation and planning algorithms for autonomous robots using 3D Scene-Graphs for live mapping onboard a Boston Dynamics Spot robot
	% \item Applying novel 3D Scene-Graph technology onboard a Spot quadruped robot to generate hierarchical maps for scene understanding %Aiming for publication of results for RSS 2024.
	% \item Developing Reinforcement Learning pipeline in Habitat simulator to train navigation and exploration policy for a Boston Dynamics Spot robot to be applied in real-world outdoor search problems
	% \item Researching the impact of state-of-the-art Counter-ATR algorithms using foundational computer vision models, such as CLIP or LLaVA
	% \item Led team of 10 interns to build a SpotMicro quadruped robot.
	% \item Programmed Large Language Model based resume ranker for interns, allowing for accelerated review of over 150 internship applicants
	% \item Developing advanced navigation and planning algorithms for autonomous robots, enhancing mapping efficiency by 30% using 3D Scene-Graphs on a Boston Dynamics Spot robot.
	% \href{https://graphex.mit.edu/sites/default/files/images/2024%20GraphEx%20Poster%20-%20Trang.pdf}{Hierarchical 3D Scene Graphs for Robot Applications}
	\item Implementing novel  \href{https://ieeexplore.ieee.org/document/10659066}{open-set 3D Scene Graph} technology on Spot quadruped robot, improving real-time scene graph construction and achieving a success rate of 71\% in mobile manipulation
	\item Leading development of Graph Reinforcement Learning algorithms in Habitat using PPO and GCNs, training robot policies for task-based navigation
	\item Managing a 4-engineer team within a 12-person, multi-organization project (U.S. Air Force, MIT, and Lincoln Laboratory), delivering autonomous vehicle swarms for Personnel Recovery
	\item Demonstrating real-world capabilities of autonomous vehicles swarms to collaborate with human supervisors and follow language-driven guidance to map, explore, and monitor an area
	\item Researched exploitation of target-recognition software using computer vision models such as CLIP, ResNet, GPT4o, and LLaVA, achieving 90\% accuracy on military classification task
	\item Developed an automated testing suite for CV models with a custom labeling UI, boosting data quality and research efficiency by 400\%
	% \item Programmed and deployed a Large Language Model-based resume ranking tool, streamlining the intern selection process and efficiently reviewing over 150 applications

	% \item Devised novel reward schemes and neural networks for RL AI Fighter Jet Agents,
	% achieving 1st place in DARPA ADT
\end{zitemize}

%====================
% EXPERIENCE B
%====================
\subsection{{Reinforcement Learning Researcher / M.S. Computer Engineering  \hfill Dec 2021 --- Dec 2022}}
\subtext{Virginia Tech \hfill Blacksburg, VA}
\begin{zitemize}
	% \item Researched Multi-Task Reinforcement Learning for Single-Agent and Multi-Agent drones to study the impact of incremental learning and catastrophic forgetting on complex navigational tasks
	% \item Created Incremental Learning with Second-Order Approximation Regularization (IL-SOAR) algorithm to mitigate catastrophic forgetting in multi-task learning which resulted in 33\% improvement over normal incremental learning
	% \item Developed \href{https://github.com/trangml/multi-task-pybullet-drones}{multi-task-pybullet-drones} simulation environment for training and testing RL agents in PyBullet, with Hydra for hyperparameter optimization, YAML-based configuration, and multi-task training examples
	% \item Conducted Multi-Task Reinforcement Learning research on drones, focusing on incremental learning and its impact on complex navigational tasks
	\item Developed the Incremental Learning with Second-Order Approximation Regularization (IL-SOAR) algorithm, enhancing multi-task learning efficiency by 33\% by mitigating catastrophic forgetting
	\item  Created a robust multi-task simulation framework in PyBullet, integrating YAML for configuration management; decreased setup time for training sessions by 50\%, enhancing overall efficiency for RL model testing.
	% Created the \href{https://github.com/trangml/multi-task-pybullet-drones}{multi-task-pybullet-drones} simulation environment in PyBullet for RL agent training, featuring Hydra for hyperparameter optimization and YAML for configuration
\end{zitemize}

%====================
% EXPERIENCE A
%====================
\subsection{{Machine Learning Engineer / DARPA ACE, Gamebreaker, etc. \hfill Dec 2019 --- Aug 2022}}
\subtext{Shield AI \hfill Alexandria, VA}
\begin{zitemize}
	% \item Trained RL agents, devised novel reward schemes, and implemented state of the art RL algorithms for government defense contracts advancing transfer learning, trustworthy AI, and complex
	% control systems, achieving 1st place in DARPA ADT.
	% \item Bootstrapped RL Testing Environment for creating low-to-high fidelity generalized transfer learning algorithms
	% to provide five different testing environments with configurable difficulties
	% \item Coded custom neural network modules for validating game balance for the DARPA Gamebreaker challenge, generating a 90\%
	% accurate win probability classifier for Starcraft II with interactive React JS dashboard
	% \item Devised novel reward schemes and neural networks for RL AI Fighter Jet Agents,
	% achieving 1st place in DARPA ADT
	\item Trained RL fighter jet agents and implemented novel AI trust capabilities, culminating in a first-place finish in DARPA's AlphaDogfight Trials
	% \item Established a versatile RL Testing Environment, facilitating the development of low-to-high fidelity transfer learning algorithms across five distinct, adjustable difficulty settings
	\item Engineered custom neural network modules for the DARPA Gamebreaker challenge, successfully developing a Starcraft II win probability classifier with 90\% accuracy, complemented by an interactive React JS dashboard

\end{zitemize}


%====================
% EXPERIENCE B
%====================
% \subsection{{Graduate Teaching Assistant / ECE 3574 Applied Software Design  \hfill Jan 2022 --- Present}}
% \subtext{Virginia Tech \hfill Blacksburg, VA}
% \begin{zitemize}
% 	\item Collaborate with Professors and TAs to formulate comprehensive software design curriculum and projects for two
% 	semesters
% 	\item Taught subject matter and assisted students with software projects for two classes with \textasciitilde 70 students in
% 	total using C++ and Qt
% \end{zitemize}

%====================
% EXPERIENCE B
%====================
% \subsection{{Senior Design Team Member / PowerHAUS \hfill Feb 2021 --- Dec 2021}}
% \subtext{Virginia Tech \hfill Blacksburg, VA}
% \begin{zitemize}
% 	\item Designed Tensorflow2 object detection image classifier and AR mobile app for monitoring smart devices with a
%     limited dataset
% 	% \item Validated safety and functionality of power electronics cartridge consisting of high-voltage systems such as a
% 	% solar panel array, high-voltage battery, and inverter prior to deployment at the Dubai Expo 2022
% 	\item Validated safety and functionality of power electronics cartridge consisting of high-voltage systems for Dubai Expo 2022
% \end{zitemize}

%====================
% EXPERIENCE C
%====================
% \subsection{{Perception Team Member / Victor Tango AutoDrive \hfill Nov 2018 --- Sep 2020}}
% \subtext{Virginia Tech \hfill Blacksburg, VA}
% \begin{zitemize}
% 	% \item Collaborated with 30+ team members on cross-discipline team to design a fully-autonomous self-driving
%     % vehicle as part of the SAE AutoDrive challenge
% 	% \item Utilized Lidar data and point cloud mapping techniques to create a function for stop sign detection using ROS, QNX, and MATLAB
% 	% \item Integrated localization and precision IMU sensor with communication network to control vehicle steering, braking,
%     % and torque
% 	% \item Played a pivotal role in a multidisciplinary team of over 30 members to engineer a fully-autonomous vehicle for the SAE AutoDrive Challenge
% 	% \item Developed a stop sign detection algorithm employing Lidar data and point cloud mapping, leveraging ROS, QNX, and MATLAB
% 	% \item Spearheaded the integration of localization and precision IMU sensors with the vehicle’s communication network to optimize steering, braking, and torque control
% 	\item Collaborated in a cross-disciplinary team of 30+ at the SAE AutoDrive Challenge, designing a fully-autonomous vehicle using ROS, QNX, and MATLAB
% 	\item Spearheaded the development of a Lidar-based stop sign detection function and integration of precision IMU sensor to improve self-driving performance
% \end{zitemize}

%====================
% EXPERIENCE D
%====================
% \subsection{{Embedded UAV Software Engineering SEPP Intern / Software Systems Group  \hfill May 2020 --- Aug 2020}}
% \subtext{Collins Aerospace \hfill Sterling, VA}
% \begin{zitemize}
% 	\item Programmed multi-camera visual navigation pipeline for a GPS-denied UAV using MATLAB Simulink and C++ on Jetson
%     TX2
% 	\item Collaborated remotely with team of two fellow interns to demonstrate vision-based autonomous landing with fiducial
% 	markers
% \end{zitemize}

%====================
% EXPERIENCE D
%====================
% \subsection{{Design Lead \& Upperclassman Advisor / Team Juvo  \hfill Sep 2018 --- Aug 2020}}
% \subtext{Virginia Tech \hfill Blacksburg, VA}
% \begin{zitemize}
% 	\item Modeled, prototyped, and 3d-printed a Wearable Mouse Band assistive device to help a disabled student use a
% 	computer
% 	\item Improved computer navigation speeds of the student user by 30\% and accurate click rate by 80\% for 2+ years
% 	of college
% \end{zitemize}

%====================
% EXPERIENCE E
%====================
%\subsection{{ROLE / PROJECT E \hfill MMM YYYY --- MMM YYYY}}
%\subtext{company E \hfill somewhere, state}
%\begin{zitemize}
%\item In lobortis libero consectetur eros vehicula, vel pellentesque quam fringilla.
%\item Ut malesuada purus at mi placerat dapibus.
%\item Suspendisse finibus massa eu nisi dictum, a imperdiet tellus convallis.
%\item Nam feugiat erat vestibulum lacus feugiat, efficitur gravida nunc imperdiet.
%\item Morbi porta lacus vitae augue luctus, a rhoncus est sagittis.
%\end{zitemize}

% Design Lead & Upperclassman Advisor, Team Juvo, Virginia Tech    Sep 2018 – Aug 2020
% Designed and built a Wearable Mouse Band to assist a disabled student in utilizing his computer
% Improved computer navigation speeds of the student user by 30% and accurate click rate by 80%
% Electrical Team Lead & Outreach Head, InVenTs Rocketry, Virginia Tech    Sep 2018 – Aug 2020
% Programmed Avionics Bay for rocket in NASA’s Space Grant Midwest High-Power Rocket Competition
% Coordinated community educational activities at local elementary school to spread STEM interest
% Space and Defense Co-op, Moog Inc.    May 2019 – Aug 2019
% Led Computer Vision research at Moog Blacksburg on Xilinx FPGA using SDSoC and MATLAB HDL Coder
% Programmed multiple CV algorithms such as a Harris Corner Detector that runs 500x faster on hardware
